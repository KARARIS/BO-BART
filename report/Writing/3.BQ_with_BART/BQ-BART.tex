% \subsection{BQ with CART}
% \label{BQ with CART}

% Using Bayesian CART, we will learn a function $f_{CART}$ and approximate:
% \begin{equation}
% 	I_f \approx \int_{x \in \mathcal{X}} f_{\mbox{CART}}(x) p(x) dx
% \label{eq:approx1}
% \end{equation}

% We denote the value of terminal node $\rho_i$ by $f_{\mbox{CART}}(\rho_i)$. 

% We further augment our tree data structure by calculating Eq.~\eqref{eq:integral}, our original integral, for the set of events that reach
% each terminal node:
% \begin{align}
% 	& \int_{x \in \rho_i} f_{\mbox{CART}}(x) p(x) dx \\
% 	&=  f_{\mbox{CART}}(\rho_i) p(\rho_i)
% 	\label{eq:approx2}
% \end{align}

% \textcolor{red}{\textit{Not so sure whether to use $q(\rho_i)$ or $p(\rho_i)$ here. In Eq.~\eqref{eq:integral} we define $p(x)$ as the distribution but I reckon think this $p$ here stands for the terminal probability which I defined as $q$ in Section~\ref{Introduction}.}} \textcolor{blue}{\textit{Let us try to use $p(x)$ throughout; see if you can update Section~\ref{Introduction} accordingly.  Actually, the details of how we calculate / store $p(x)$ don't really belong there. They should be in Section 3.3}}

% Assuming we have $b$ terminal nodes and having Eq.~\eqref{eq:approx2}, Eq.~\eqref{eq:approx1} becomes:
% \begin{align}
% %	& \int_{x \in \mathcal{X}} f(x) p(x) dx \\ 
% %	\int_{x \in \mathcal{X}} f_{\mbox{CART}}(x) p(x) dx \\ 
% 	\sum_{i=1}^b \int_{x \in \rho_i} f_{\mbox{CART}}(x) p(x) dx = \sum_{i=1}^b f_{\mbox{CART}}(\rho_i) p(\rho_i)
% \label{CART sum}
% \end{align}

% Now given a set of trees $T_1, \ldots, T_m$ drawn from the Bayesian CART posterior, we have
% functions $f_{\mbox{CART}}^{(1)}, \ldots, f_{\mbox{CART}}^{(m)}$, each with its own set of terminal
% nodes $\rho_i^{j}$, $i=1, \ldots, b_j$. We use these trees to approximate
% Eq.~\eqref{eq:integral}:
% \begin{align}
% 	I_f = \int f(x)p(x) dx \approx \frac{1}{m} \sum_{j=1}^m \sum_{i=1}^{b_j} f_{\mbox{CART}}^j(\rho_i^j) p(\rho_i^j)
% \end{align}

% After fitting CART, we calculate $I_f$ for each tree drawn from the posterior.
% If we draw $P$ samples, we obtain $I_f^{(1)}, \ldots, I_f^{(P)}$, allowing us to summarise
% the mean and variance as:
% \begin{equation}
% 	\mbox{E}[I_f] = \frac{1}{n} \sum_p I_f^{(p)}
% \end{equation}
% \begin{equation}
% 	\mbox{Var}[I_f] = \frac{1}{n} \sum_p (I_f^{(p)} - \mbox{E}[I_f])^2
% \end{equation}

\subsection{BQ with BART}
\label{BQ with BART}
We will learn a function $f_{BART}$ and approximate:
\begin{equation}
	I_f \approx \int_{x \in \mathcal{X}} f_{\mbox{BART}}(x) p(x) dx
\label{eq:approx1}
\end{equation}
We denote the value of terminal node $\rho_i$ by $f_{\mbox{BART}}(\rho_i)$. 

Depending on the tree structure, we have $f^j_{k\mbox{BART}}$, which stands for the $k^{th}$ tree in the $j^{th}$ posterior draw. And the Eq.~\eqref{eq:single tree approximation} integral approximation given by one set of trees becomes
\begin{align}
	I_f \approx \frac{1}{ntree}\sum_{k=1}^{ntree}\sum_{i=1}^{b_j} f_{k\mbox{BART}}(\rho_i^k) p(\rho_i^k)
\end{align}
where $ntree$ is the pre-set value in the BART model as the number of posterior draws.

After fitting BART, we calculate $I_f$ for each tree drawn from the posterior.
If we draw $P$ samples, we obtain $I_f^{(1)}, \ldots, I_f^{(P)}$, allowing us to summarise
the mean and variance as:
\begin{equation}
	\mbox{E}[I_f] = \frac{1}{n} \sum_p I_f^{(p)}
\end{equation}
\begin{equation}
	\mbox{Var}[I_f] = \frac{1}{n} \sum_p (I_f^{(p)} - \mbox{E}[I_f])^2
\end{equation}

\subsection{Probability Associated with Terminal Nodes}
\label{p}

If we assume a uniform prior $p(x)$ this will allow us to conveniently reuse the calculations that are already used for sampling. According to Hugh A. Chipman et al., at each node, all predictors are equally likely to be effective.

This means that calculating $p(x)$ we mentioned in Section~\ref{Introduction} should
be simple: reuse the calculation that is built into BART for the uniform specification of $p_{\mbox{RULE}}$.

To find the probability, based on the uniform distribution we used, we find the conditional probability on the location of each cutting point in the total interval.
